\documentclass{article}
\begin{document}
The project involves fluid simulation, initially based on the methods outlined in Robert Bridson and Matthias Muller-Fischer's 2007 Siggraph course.
Liquid fluids, and to a lesser extend gases, can be considered incompressible in most simulations. The behaviour of incompressible fluids is given by the Navier-Stokes equation:

[Navier-Stokes equation]

This is simplified by breaking the equation down into three parts: the advection, body, and projection equations, and applying their respective processes advect, body and project in sequence for each time step


When applied to the entire fluid, the advection equation enforces the conservation of energy (the fuild mass undergoes no net acceleration without the application of a net force). The advect( process applies the velocity field to the quantity q for a time dt. This quantity may be velocity itself, pressure, or another quantity relevant to the particular fluid (such as temperature for smoke simulation).
Since in truth the fluid is acted on by the net force of gravity, the body process simply adds gdt to the velocity field (all the members of the grid).
The incompressibility of the fluid is enforced by the 

Seven Point Lagrangian Matrix
MIC(0) preconditioner
\end{document}

